Se desarrolló un algoritmo capaz de seguir a un objeto frente a oclusiones parciales y totales, cambios en la iluminación y cambios bruscos en su trayectoria. Se logró minimizar la probabilidad de cometer error de tipo I, tipo II y tipo III mencionados anteriormente utilizando una combinación de varios algoritmos y técnicas distintas, entre ellas: Sparse Optical Flow de Lucas Kanade para la detección de movimiento; el algoritmo de Shi-Tomasi para la búsqueda de esquinas características de un objeto; filtros de Kalman para lograr estimar la trayectoria frente a una oclusión total; la utilización de filtros de distribución de color para detectar al objeto a seguir según su distribución de color y la probabilidad de ubicar esta distribución en la pantalla junto a filtros de color utilizando el espacio de colores CIE-Lab para el enmascaramiento de las zonas sin interés de la pantalla; y finalmente filtros de correlación adaptativos para desambiguar el seguimiento de un objeto solamente tomando como característica su color, lo cual reduce significativamente la probabilidad de seguir a un objeto de mismo color que el seleccionado.




%Se logró el seguimiento de un objeto especificado por el usuario en base a cualidades distintivas del mismo, %siendo estas sus bordes y su color o distribución de color, al igual que por su movimiento y correlación. %Contando con una predicción certera en el caso de que se encuentre el objeto en pantalla al igual que la %situación de que desaparezca de pantalla, aunque la varianza de la predicción aumentará cuanto mas tiempo no sea %detectado, con la capacidad encontrar nuevamente a este. Tambien cuenta con la capacidad de en el caso de perder %al objeto de interés por otro objeto de cualidades similares, detectar este error y corregirlo volviendo al %objeto original.\\
%La elección de los algoritmos y filtros utilizados son sinérgicos dado que estos se complementan entre sí, por %ejemplo el filtro de color proporciona bordes definidos para el objeto a seguir, que es ideal para la obtención %de bordes de Shi-Tomasi