Se logró seguimiento de un objeto especificado por el usuario en base a cualidades distintivas del mismo, siendo estas sus bordes y su color o distribución de color, al igual que por su movimiento y correlación. Contando con una predicción certera en el caso de que se encuentre el objeto en pantalla al igual que la situación de que desaparezca de pantalla, aunque la varianza de la predicción aumentará cuanto mas tiempo no sea detectado, con la capacidad encontrar nuevamente a este. Tambien cuenta con la capacidad de en el caso de perder al objeto de interés por otro objeto de cualidades similares, detectar este error y corregirlo volviendo al objeto original.\\
La elección de los algoritmos y filtros utilizados son sinérgicos dado que estos se complementan entre sí, por ejemplo el filtro de color proporciona bordes definidos para el objeto a seguir, que es ideal para la obtención de bordes de Shi-Tomasi