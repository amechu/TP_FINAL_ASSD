Algunas mejoras que pueden ser implementadas en futuros desarrollos son:
\begin{itemize}
\item Trackeado multiobjeto: \\
Da la opción al usuario de trackear mas de un objeto en la pantalla, utilizando filtros de color diferenciados por cada objeto.
\item Filtro de Kalman donde la matriz de covarianza de medición no sea constante: \\
En el modelo utilizado actualmente la matriz de covarianza es constante, esto asume que la presición de  medición es fija. Esto último no suele ser verdad en la realidad. Una mejora es la introducción de modificaciones en la matriz de covarianza de manera dinámica así el filtro proporcionaría mejores predicciones. 


\item Filtros de correlación:\\
El uso de filtros de correlación tomando como cuadro inicial la selección del usuario se calcula  la correlación con toda la pantalla entre cuadros, donde la correlación sea mayor será mas probable la coincidencia.

\item Invarianza ante cambios de iluminación:\\
El cambio de iluminación impacta directamente sobre el color percibido. Esto afecta directamente a nuestra técnica de filtrado por color. Es por eso que se desean explorar formas de hacer más robusto al sistema ante estos cambios.

\item Filtro de sensibilidad Multicolor: \\
Dado tres objetos $A$, $B$ y $C$, donde $A$ cuenta con 2 colores distintivos, siendo estos rojo y azul, $B$ un objeto rojo y $C$ uno azul, lo que propone este filtro es la capacidad de diferenciar al objeto a seguir por mas de un color distintivo, cosa de que la interferencia de $B$ o $C$ en el camino de $A$ no comprometa su seguimiento debido a que ni $B$ ni $C$ posee la combinación de colores de $A$.
\item Un mejor modelado de las ecuaciones dinámicas del filtro de Kalman para incluir la aceleración como variable de estado: \\
El modelo utilizado actualmente consta de un movimiento rectilíneo uniforme bidimensional. La inclusión de la aceleración proporcionaría una mejor predicción.
\end{itemize}